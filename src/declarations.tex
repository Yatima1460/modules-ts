%!TEX root = std.tex
\setcounter{chapter}{9}
\rSec0[dcl.dcl]{Declarations}%
%\indextext{declaration|(}

%gram: \rSec1[gram.dcl]{Declarations}
%gram:

\noindent
Add a new alternative to \term{declaration} in paragraph 10/1 as follows
\begin{std.txt}
  \begin{bnf}
    \nonterminal{declaration}:\br
      block-declaration\br
      nodeclspec-function-declaration\br
      function-definition\br
      template-declaration\br
      explicit-instantiation\br
      explicit-specialization\br
      linkage-specification\br
      namespace-definition\br
      empty-declaration\br
      attribute-declaration\br
      \color{addclr}
      \added{export-declaration\br
      module-import-declaration\br
      proclaimed-ownership-declaration}
  \end{bnf}
  \end{std.txt}


\setcounter{section}{0}
\rSec1[dcl.spec]{Specifiers}%

\setcounter{subsection}{1}
\rSec2[dcl.fct.spec]{Function specifiers}%

\noindent
Add a new paragraph 10.1.2/7 as follows:
\begin{std.txt}
  \color{addclr}
  \resetalinea[6]
  \alinea
  \added{An exported inline function shall be defined in the same translation
  unit containing its export declaration.
%  That semantic property
%  is a reachable semantic property (\ref{dcl.module.reach}) even if
%  the definition is not exported (\ref{dcl.module.interface}).
  \enternote
  There is no restriction on the linkage (or absence thereof)
  of entities that the function body of an exported inline function
  can reference. A constexpr function (10.1.5) is implicitly inline.
  \exitnote}
\end{std.txt}


\setcounter{subsection}{5}
\rSec6[dcl.inline]{The \tcode{inline} specifier}%

\noindent
Modify paragraph 10.1.6/6 as follows
\begin{std.txt}
  \resetalinea[5]
  \alinea
  \added{Some definition for}
  \removed{A}\added{a}n inline function or variable shall be 
  \removed{defined}\added{reachable} in every translation 
  unit in which it is odr-used and \added{the function or variable} shall have exactly 
  the same definition in every case (\ref{basic.link}).
  \enternote
  A call to the inline function or a use of 
  the inline variable may be encountered before its definition appears 
  in the translation unit.
  \exitnote
  If the definition of a function or variable appears in a translation unit 
  before its first declaration as inline, the program is ill-formed. If a 
  function or variable with external \added{or module} linkage is 
  \removed{declared}\added{reachable via an} 
  inline \added{declaration} in one translation 
  unit, it shall be \removed{declared}\added{reachable via an} inline
  \added{declaration} in all translation units in which it 
  \removed{appears}\added{is reachable}; 
  no diagnostic is required. An inline function or variable with 
  external \added{or module} linkage 
  shall have the same address in all translation units.
  \enternote
  A \tcode{static} local variable in an inline function with external 
  \added{or module} linkage always refers to the same object. A type 
  defined within the body 
  of an inline function with external \added{or module} linkage is the 
  same type in every translation unit.
  \exitnote
\end{std.txt}




\setcounter{section}{2}
\rSec1[basic.namespace]{Namespaces}%
%\indextext{namespaces|(}

\noindent
Modify paragraph 10.3/1 as follows:
\begin{std.txt}
  \resetalinea[0]
  \alinea
  A namespace is an optionally-named declarative region. The name of a
  namespace can be used to access entities declared in that namespace;
  that is, the members of the namespace. Unlike other declarative
  regions, the definition of a namespace can be split over several
  parts of one or more translation units.
  \added{A namespace with external linkage is always 
    exported regardless of whether any of its \term{namespace-definition}{s} is
    introduced by \tcode{export}.
    \enternote
    There is no way to define a namespace with module linkage.
    \exitnote
  }\color{addclr}
  \begin{example}
  \begin{codeblock}
    export module M;
    namespace N {    // N has external linkage and is exported
    }
  \end{codeblock}
  \end{example}
\end{std.txt}

\setcounter{subsection}{2}
\rSec2[namespace.udecl]{The \tcode{using} declaration}

Modify paragraph 10.3.3/1 as follows:
\begin{std.txt}
  \resetalinea[0]
  \alinea
  Each \grammarterm{using-declarator} in a \grammarterm{using-declaration}
  introduces a set of declarations \added{and citations} into the declarative 
  region in which the
  \grammarterm{using-declaration} appears.  The set of declarations
  \added{and citations}
   introduced by
  the \grammarterm{using-declarator} is found by performing qualified name
  lookup (6.4.3, 13.2) for the name in the \grammarterm{using-declarator},
  excluding functions that are hidden as described below.  If the
  \grammarterm{using-declarator} does not name a constructor, the 
  \grammarterm{unqualified-id} is declared in the declarative region in which
  the \grammarterm{using-declaration} appears as a synonym for each
  declaration \added{or citation} 
  introduced by the \grammarterm{using-declarator}. [...]
\end{std.txt}


\noindent
Add a new subclause 10.7 titled ``\textbf{Modules}'' as follows:

\setcounter{section}{6}
\rSec1[dcl.module]{Modules}%

\rSec2[dcl.module.unit]{Module units and purviews}

\begin{std.txt}\color{addclr}
  \begin{bnf}\color{addclr}
    \nonterminal{module-declaration}:\br
       \terminal{export}\opt {} \terminal{module} module-name attribute-specifier-seq\opt \terminal{;}
  \end{bnf}
  
  \begin{bnf}\color{addclr}
    \nonterminal{module-name}:\br
      module-name-qualifier-seq\opt identifier
  \end{bnf}
  
  \begin{bnf}\color{addclr}
    \nonterminal{module-name-qualifier-seq}:\br
      module-name-qualifier  \terminal{.}\br
      module-name-qualifier-seq identifier \terminal{.}
  \end{bnf}
  
  \begin{bnf}\color{addclr}
    \nonterminal{module-name-qualifier}:\br
      identifier
  \end{bnf}
  
  \resetalinea[0]
\alinea A \term{module unit} is a translation unit that contains
a \grammarterm{module-declaration}. A \term{named module} is the
collection of module units with the same \grammarterm{module-name}.
A translation unit shall not contain more than
one \grammarterm{module-declaration}. A \grammarterm{module-name}
has external linkage but cannot be found by name lookup.

\alinea
A \term{module interface unit} is a module unit whose
\grammarterm{module-declaration} contains the \texttt{export} keyword;
any other module unit is a \term{module implementation unit}.  A named
module shall contain exactly one module interface unit.

\alinea
A \term{module unit purview} starts at the \grammarterm{module-declaration}
and extends to the end of the translation unit.
The \term{purview} of a named module \tcode{M} is the set of module unit purviews
of \tcode{M}'s module units. 

\alinea
The \term{global module} is the collection of all declarations
not in the purview of any module.   By
extension, such declarations  are said to be in the purview of the
global module. 
\enternote
The global module has no name, no module interface unit, and is not
  introduced by any 
  \grammarterm{module-declaration}.
\exitnote

\alinea
  A \term{module} is either a named module or the global module.
  A \grammarterm{proclaimed-ownership-declaration} is
  \term{attached} to the module it nominates; any other declaration
  is attached to the module in whose purview it appears.

  \alinea
For a namespace-scope declaration $D$ of an entity (other than a
namespace), if $D$ is within a \grammarterm{proclaimed-ownership-declaration}
for a module $X$, the entity is said to be \term{owned} by $X$.
Otherwise, if $D$ is the first declaration of that entity, then that entity is said
to be \term{owned} by the module in whose purview $D$ appears. 

  \alinea
  If a declaration attached to some module
  matches (according to the redeclaration rules) a reachable declaration
  from a different module, 
  the program is ill-formed.
  \begin{example}
    \begin{codeblock}
      // module interface of M
      int f();            // \#1
      int g();            // \#2, owned by the global module
      export module M;
      export using ::f;   // OK: does not declare an entity
      int g();            // error: matches \#2, but appears in the purview of M
      export int h();     // \#3
      export int k();     // \#4

      // other translation unit
      import M;
      static int h();     // error: matches \#3
      int k();            // error: matches \#4
    \end{codeblock}
  \end{example}

  \alinea
  The subgraph of the abstract semantics graph $G$ of a module $M$
  generated by the nodes of $G$, excluding those introducing names
  with internal linkage, is available to name lookup in the purview of every
  module implementation unit of $M$.
  The declsets made available by the \grammarterm{module-import-declaration}{s}
   in the purview of the module
  interface unit of $M$ are also available to name lookup in the purview of all 
  module implementation units of $M$.
\end{std.txt}

\rSec2[dcl.module.interface]{Export declaration}%

\begin{std.txt}\color{addclr}
  \begin{bnf}\color{addclr}
    \nonterminal{export-declaration}:\br
      \terminal{export} declaration\br 
      \terminal{export} \terminal{\{} declaration-seq${}_{opt}$ \terminal{\}} 
  \end{bnf}

  \resetalinea[0]
  \alinea 
  An \grammarterm{export-declaration} shall only appear
  at namespace scope and only in the purview of a module interface unit. 
  An \grammarterm{export-declaration} shall not appear directly
  or indirectly within an unnamed namespace. 
  An \grammarterm{exported-declaration}
   has the declarative effects of its \grammarterm{declaration}
  or its \grammarterm{declaration-seq} (if any).
  An \grammarterm{export-declaration} does not
  establish a scope and shall not contain more than one
  \tcode{export} keyword.
  The \term{interface} of a module \tcode{M} is the set of all
  \grammarterm{export-declaration}{s} in its purview.

  \alinea
  In an \grammarterm{export-declaration} of the form
  \begin{grammar}
        @\tcode{export}@ @\term{declaration}@
  \end{grammar}
  the \grammarterm{declaration} shall be a \grammarterm{module-import-declaration},
  or it shall declare at least one name,
  and if that declaration declares an entity, the 
  \grammarterm{decl-specifier-seq} (if any) of the 
  \grammarterm{declaration} shall not contain \tcode{static}.
  The \grammarterm{declaration} shall not be an
  \grammarterm{unnamed-namespace-definition}
  or a \grammarterm{proclaimed-ownership-declaration}.
  \begin{example}
    \begin{codeblock}
      export int x;             // error: not in the purview of a module interface unit
      export module M;
      namespace {
        export int a;           // error: export within unnamed namespace
      }
      export static int b;      // error: b explicitly declared static.
      export int f();           // OK
      export namespace N { }    // OK
      export using namespace N; // error: does not declare a name
    \end{codeblock}
  \end{example}

  If the \grammarterm{declaration} is a \grammarterm{using-declaration}
  (\ref{namespace.udecl}), any entity to which the
  \grammarterm{using-declarator} ultimately refers shall have been introduced
  with a name having external linkage.
  \begin{example}
    \begin{codeblock}
      int f()               // f has external linkage
      export module M;
      export using ::f;     // OK
      struct S;
      export using ::S;     // error: S has module linkage
      namespace N {
        int h();
        static int h(int);  // \#1
      }
      export using N::h;    // error: \#1 has internal linkage
    \end{codeblock}
  \end{example}


  \enternote
  Names introduced by \tcode{typedef} declarations are not so constrained.
  \begin{example}
    \begin{codeblock}
      export module M;
      struct S;
      export using T = S;   // OK: exports name T denoting type S
    \end{codeblock}
  \end{example}
  \exitnote

  \alinea

  An \grammarterm{export-declaration} of the form
  \begin{grammar}
    @\tcode{export}@ @\tcode{\{} \term{declaration-seq${}_{opt}$} \tcode{\}}@ 
  \end{grammar}
  is equivalent to a sequence of declarations formed by prefixing each
  \grammarterm{declaration} of the \grammarterm{declaration-seq} (if any) with
  \tcode{export}.

  \alinea
  A namespace-scope or a class-scope declaration lexically
  contained in an \grammarterm{export-declaration}, as well as the
  entities and the names it introduces are said to be \term{exported}.
  The exported declarations in the interface of a module are reachable from
  any translation unit importing that module.  
  \enternote
  Exported names have either external linkage or no linkage; see \ref{basic.link}
  \exitnote
%  The declarations introducing names with linkage
%  other than internal linkage introduced or made reachable
%   (via an \grammarterm{import-declaration})
%   in the purview of the module
%  interface unit of a 
%  module \tcode{M} are reachable from the purview of all module
%  implementation units
%  of \tcode{M}. 
  \begin{example}
    \begin{codeblock}
      // Interface unit of M
      export module M;
      export struct X {
        void f();
        struct Y { };
      };

      namespace {
        struct S { };
      }
      export void f(S);    // OK
      struct T { };
      export T id(T);      // OK
  
      export struct A;     // A exported as incomplete

      export auto rootFinder(double a) {
        return [=](double x) { return (x + a/x)/2; };
      }

      export const int n = 5; // OK: n has external linkage
  
      // Implementation unit of M
      module M;
      struct A {
        int value;
      };
  
      // main program
      import M;
      int main() {
        X{}.f();                // OK: X and X::f are exported
        X::Y y;                 // OK: X::Y is exported as a complete type
        auto f = rootFinder(2); // OK
        return A{45}.value;     // error: A is incomplete
      }
      \end{codeblock}
  \end{example}


  
  \alinea
  \enternote
  Redeclaring a name in an \grammarterm{export-declaration}
  cannot change the linkage of the name (10.1.1).
  \begin{example}
  \begin{codeblock}
    // Interface unit of M
    export module M;
    static int f();             // \#1
    export int f();             // error: \#1 gives internal linkage
    struct S;                   // \#2
    export struct S;            // error: \#2 gives module linkage
    namespace {
      namespace N {
        extern int x;           // \#3
      }
    }
    export int N::x;            // error: \#3 gives internal linkage
  \end{codeblock}
  \end{example}
  \exitnote
%% If the
%% \term{export-declaration} introduces a function template or a variable
%% template then the type of the corresponding current instantiation shall
%% contain only types with external linkage.  If the
%% \term{export-declaration} introduces a template alias then the aliased
%% type shall have external linkage.  If the
%% \term{export-declaration} defines a class template, then all
%% non-internal members of the corresponding current instantiation shall
%% contain only types with external linkage.

%%   \alinea
%%   In a \grammarterm{exported-fragment-group},
%%    each \grammarterm{fragment} is processed
%%   as an exported declaration.

  \alinea
  Declarations in an exported \grammarterm{namespace-definition} 
  or in an exported \grammarterm{linkage-specification} (10.5)
  are
  implicitly exported and subject to the rules of exported declarations.
  \begin{example}
    \begin{codeblock}
    export module M;
    export namespace N {
      int x;                    // OK
      static_assert(1 == 1);    // error: does not declare a name
    }
    \end{codeblock}
  \end{example}
\end{std.txt}


\rSec2[dcl.module.import]{Import declaration}%

\begin{std.txt}\color{addclr}
  \begin{bnf}\color{addclr}
    \nonterminal{module-import-declaration}:\br
      \terminal{import} module-name attribute-specifier-seq\opt {} \terminal{;}
  \end{bnf}

  \resetalinea[0]
  \alinea
  A \grammarterm{module-import-declaration} shall appear only at
  global scope, and not in a \grammarterm{linkage-specification}
  or \grammarterm{proclaimed-ownership-declaration}.
  A \grammarterm{module-import-declaration} nominating a module $M$
  makes every citation and every exported declaration from the 
  abstract semantics graph of $M$ available, as a citation,
  %makes exported declarations from the interface of the nominated module visible
  to name lookup in
  the current translation unit, in the same namespaces and contexts
  as in $M$.
  % the nominated module.
  A \term{citation} for a declaration attached to a module $M$ is a pair of
  $M$ and the corresponding declset from the abstract semantics graph 
  of $M$. 
%  The \grammarterm{attribute-specifier-seq} appartain to the nominated module.
  \enternote
  The declarations in the declsets and the entities denoted by the declsets
  are not redeclared in the translation unit
    containing the \grammarterm{module-import-declaration}.
  \exitnote
  \begin{example}
  \begin{Program}
     // Interface unit of M
     export module M;
     export namespace N {
        struct A { };
     }
     namespace N {
        struct B { };
        export struct C {
           friend void f(C) { }  // exported, visible only through argument-dependent lookup
        };
     }

     // Translation unit 2
     import M;
     N::A a { };                // OK.
     N::B b { };                // error: `B' not found in N.
     void h(N::C c) {
        f(c);                   // OK: `N::f' found via argument-dependent lookup
        N::f(c);                // error: `f' not found via qualified lookup in N.
     }
  \end{Program}
  \end{example}

  \alinea
  A module \tcode{M1} \term{has a dependency} on a module \tcode{M2}
  if any module unit of \tcode{M1} contains a \grammarterm{module-import-declaration}
  nominating \tcode{M2}. A module shall not have a dependency on
  itself.
  \begin{example}
  \begin{codeblock}
     module M;
     import M;          // error: cannot import M in its own unit.
  \end{codeblock} 
  \end{example}

  \alinea 
  A module \tcode{M1} \term{has an interface dependency} on a module
  \tcode{M2} if the module interface of \tcode{M1} contains a
  \grammarterm{module-import-declaration} nominating \tcode{M2},
  or if there exists a module \tcode{M3} such that \tcode{M1} has an
  interface dependency on \tcode{M3} and \tcode{M3} has an interface dependency
  on \tcode{M2}.  A module
  shall not have an interface dependency on itself.
  \begin{example}
  \begin{Program}
     // Interface unit of M1
     export module M1;
     import M2;

     // Interface unit of M2
     export module M2;
     import M3;

     // Interface unit of M3
     export module M3;
     import M1;         // error: cyclic interface dependency M3 -> M1 -> M2 -> M3
  \end{Program}
  \end{example}

  \alinea
  A translation unit has an interface dependency on a module \tcode{M} if it is
  a module implementation unit of \tcode{M}, or if it contains a
  \grammarterm{module-import-declaration} nominating \tcode{M}, or if it has
  an interface dependency on a module that has an interface dependency on \tcode{M}.
\end{std.txt}


\rSec2[dcl.module.export]{Module exportation}%

\begin{std.txt}\color{addclr}
  \resetalinea[0]
  \alinea
  An exported
   \grammarterm{module-import-declaration} nominating a module \tcode{M2}
  in the purview of a module interface unit of a module \tcode{M} 
  makes all exported names
  of \tcode{M2} visible to any translation unit importing \tcode{M}, as if
  that translation unit also contains a \grammarterm{module-import-declaration}
  nominating \tcode{M2}.
  \enternote
  A module interface unit (for a module \tcode{M}) containing a non-exported
    \grammarterm{module-import-declaration} does not make the imported
    names transitively visible to translation units importing the module
    \tcode{M}.
  \exitnote
  % Wording requested by Hubert Tong on December 17, 2017.
  In addition to its usual semantics, a 
  \grammarterm{module-import-declaration} nominating a module $M$ with 
  a module interface unit containing one or more exported
  \grammarterm{module-import-declaration}{s} also behaves as if
  it nominates each module nominated by an exported
  \grammarterm{module-import-declaration} in $M$; this may in turn
  lead it to be considered to nominate yet additional modules.
\end{std.txt}


\rSec2[dcl.module.proclaim]{Proclaimed ownership declaration}%

\begin{std.txt}\color{addclr}
  \begin{bnf}\color{addclr}
    \nonterminal{proclaimed-ownership-declaration}:\br
      \terminal{extern} \terminal{module} module-name \terminal{:} declaration
 \end{bnf}

 \resetalinea[0]
  \alinea
  A \grammarterm{proclaimed-ownership-declaration} shall only appear
  at namespace scope. 
  It shall not appear directly or indirectly within an unnamed namespace.
  A \grammarterm{proclaimed-ownership-declaration} has the declarative effects
  of its \grammarterm{declaration}.
  The \grammarterm{declaration} shall declare at 
  least one name, and the \grammarterm{decl-specifier-seq} (if any)
  of the \grammarterm{declaration} shall not contain \tcode{static}.
  The \grammarterm{declaration} shall not be a \grammarterm{namespace-definition},
  an \grammarterm{export-declaration},
   or a \grammarterm{proclaimed-ownership-declaration}.
  The \grammarterm{declaration} shall not be a defining declaration
  (\ref{basic.def}).
  A \grammarterm{proclaimed-ownership-declaration} nominating a module $M$
  shall not appear in the purview of $M$.

  \alinea
  A \grammarterm{proclaimed-ownership-declaration} asserts that the entities
  introduced by the declaration are exported by the nominated module. 
  \enternote
  A \grammarterm{proclaimed-ownership-declaration} may be used to break
  circular dependencies between two modules (in possibly too finely
  designed components.)
  \begin{example}
    \begin{codeblock}
    // TU 1
    export module Ty;
    extern module Sym: struct Symbol;
    export struct Type {
      Symbol* decl;
      // ...
    };

    // TU 2
    export module Sym;
    extern module Ty: struct Type;
    export struct Symbol {
      const char* name;
      const Type* type;
      // ...
    };
    \end{codeblock}
  \end{example}
  \exitnote
  
  \alinea 
  The program is ill-formed, no diagnostic required, if the
  nominated module in the  
  \grammarterm{proclaimed-ownership-declaration} does not export the entities
  introduced by the declaration. 
\end{std.txt}

\rSec2[dcl.module.reach]{Reachability}

\begin{std.txt}\color{addclr}
  \resetalinea[0]
  \alinea
  When declarations from the abstract semantics graph of a module $M$
   are made available 
  to name lookup in another translation unit $TU$, it is necessary to
  determine the interpretation of the names they introduce and their
  semantic properties. Except as noted below, the 
  \term{reachable semantic properties} of declset $D$
  (or of the entity, if any, denoted by that declset)
  of the abstract semantic graph of $M$ from $TU$ are 
  \begin{itemize}
    \item if $D$ contains at least one exported declaration, 
    the semantic properties cumulatively obtained in the context
    of the exported declaration (\ref{dcl.module.interface}) 
    members of $D$ in the module interface
    unit of $M$. Furthermore, if $D$ denotes an inline function,
    the property that the inline function has a definition
    (\ref{dcl.fct.spec}) is a reachable semantic property, even 
    if that definition is not exported. Otherwise,
    \item the semantic properties cumulatively obtained in the context of all
    declaration members of $D$ in the module interface unit of $M$.
  \end{itemize}
%  Given a declset $D$ for an exported name from a module $M$,
%  the \term{reachable semantic property} of any declaration in that declset
%  (or of the entity, if any, designated by that declset) from a translation unit
%%   For a given declaration $D$ of a name $X$ exported from a module $M$, the
%%  \term{reachable semantic properties} of $D$ from a translation unit
%  containing a \grammarterm{module-import-declaration} nominating $M$ are
%  restricted to 
%  the semantic properties cumulatively obtained in 
%  the context of the exported declaration members of $D$ in $M$.
  \enternote
  These reachable semantic properties include type completeness, 
  type definitions, initializers,
  default arguments of functions or template declarations, attributes,
  visibility to normal lookup, entities that are direct targets of edges 
  emanating from $D$ in the abstract semantics graph of $M$, etc.
  Since default arguments are evaluated in the context of the call expression,
  reachable semantic properties of the corresponding parameter types apply in 
  that context. 
  \begin{example}
    \begin{codeblock}
      // TU 1
      struct F { int f { 42 }; };
      export module M;
      export using T = F;
      export struct A { int i; };
      export int f(int, A = { 3 });

      export struct B;      // exported as incomplete type
      struct B {            // definition not exported
        operator int();
      };
      export int g(B = B{});
      export int h(int = B{}); // \#1

      export struct S {
        static constexpr int v(int);
      };

      export S j();       // S attendant entity of j()
      constexpr int S::v(int x) { return 2 * x; }


      // TU 2
      import M;
      int main() {
        T t { };             // OK: reachable semantic properties of T include completeness. 
        auto x = f(42);      // OK: default argument A\{3\} evaluated here.
        auto y = h();        // OK: completeness of B only checked at \#1.
        auto z = g();        // error: parameter type incomplete here.
        constexpr auto a = decltype(j())::v(3); // OK: S::v defined 
                            // in the abstract semantics graph of M (\ref{dcl.fct.spec})
      }
    \end{codeblock}
  \end{example}
  \exitnote

  \alinea
  Within a module interface unit, it is necessary to determine that the
  declarations being exported collectively present a coherent view of 
  the semantic properties of the entities they reference.  This determination
  is based on the semantic properties of attendant entities.
  \enternote
  The reachable semantics properties of an entity, the declarations of which
  are made available via a \grammarterm{module-import-declaration}, are
  determined by its owning module and are unaffected by the importing module.
  \begin{example}
    \begin{codeblock}
      // module interface of M1
      export module M1;
      export struct S { };

      // module interface of M2
      import M1;
      export module M2;
      export S f();       // \#1
      export S* g();      // \#2

      // elsewhere
      import M2;
      auto x = f();       // OK: completeness of S obtained at \#1
      auto y = *g();      // OK: completeness of S obtained at \#2
    \end{codeblock}
  \end{example}
  \exitnote
  
  For each declaration $D$ exported from the module interface unit of a module $M$,
   there is a set of zero or more
  \term{attendant entities} defined as follows:
  \begin{itemize}
    \item If $D$ is a type alias declaration, then the attendant entities
    of $D$ are those determined by the aliased type at the point of the 
    declaration $D$.

    \item If $D$ is a \grammarterm{using-declaration}, the set of attendant entities
    is the union of the sets of attendant entities of the declarations introduced by
    $D$ at the point of the declaration.

    \item If $D$ is a template declaration, the set of attendant entities is 
    the union of the set of attendant entities 
    of the declaration being parameterized, the set of attendant entities
    determined by the default type template arguments (if any), 
    and the set consisting of 
    the entities (if any) designated by the default template template argument,
    the default non-type template arguments (if any).

    \item if $D$ has a type $T$, the set of attendant entities is the set of 
    attendant entities determined by $T$ at the point of declaration.

    \item Otherwise, the set of attendant entities is empty.
  \end{itemize}

  The \term{set of attendant entities determined by} a type $T$ is defined as follows
  (exactly one of these cases matches):
  \begin{itemize}
    \item If $T$ is a fundamental type, then
    the set of attendant entities is empty.

  %  \item If $T$ is a dependent type denoted by a \grammarterm{typename-specifier}
  %  where the \grammarterm{nested-name-specifier} itself denotes a dependent type,
  %  then the set of attendant entities is the set of attendant entities determined by
  %  that type.
    \item If $T$ is a member of an unknown specialization, the set of
    attendant entities is the set of attendant entities determined by that
    unknown specialization.

    \item If $T$ is a class type owned by $M$, the set of attendant entities includes
    $T$ itself, the union of the sets of the attendant entities determined
    by its direct base classes owned by $M$, the sets of the
    attendant entities of its data members, static data member templates,
    member functions, member function templates, 
    the function parameters of its constructors and constructor templates.  
    Furthermore, if $T$ is a 
    class template specialization, the set of attendant entities also 
    includes: the class template if it is owned by $M$,
    the union of the sets of attendant entities determined by the type
    template-arguments, the sets of the attendant entities of the templates
    used as template template-arguments, the sets of the attendant entities
    determined by the types of the non-type template-arguments.

    \item If $T$ is an enumeration type owned by $M$, 
    the set of attendant entities is the singleton $\{T\}$.

    \item If $T$ is a reference to $U$, or a pointer to $U$, or an array of $U$, 
    the set of attendant entities is the set of attendant entities determined by $U$.

    \item If $T$ is a function type, the set of attendant entities is the
    union of the set of attendant entities determined by the function
    parameter types and the return type.

    \item if $T$ is a pointer to data member of class $X$, the set of attendant
    entities is the union of the set of attendant entities of the member type
    and the set of attendant entities determined by $X$.
    
    \item If $T$ is a pointer to member function type of a class $X$, the
    set of attendant entities is the union of the set of attendant entities
     determined by $X$ and the set of attendant entities determined by
    the function type.
    
    \item Otherwise, the set of attendant entities is empty.
  \end{itemize}

  If a class template $X$ is an attendant entity, then its reachable semantic
  properties include all the declarations of the primary class template,
  its partial specializations, and its explicit specializations in the
  containing module interface unit.
  If a complete class type $X$ is an attendant entity, then its reachable
  semantic properties include the declarations of its nested types but
  not the definitions of the types denoted by those members
   unless those definitions are exported.
  Furthermore, if $X$ is an attendant entity of an 
  exported declaration $D$, then its reachable semantic properties are 
  restricted to those defined by the exported declarations of $X$ 
  (if $X$ is introduced by an exported declaration), or by 
  the semantic properties of $X$ available at the point of the declaration $D$.
  \enternote
  If $X$ is a complete class type that is an attendant entity, its nested types
  (including nested enumerations and associated enumerators) 
  and member class templates
  are not considered attendant entities unless they are determined attendant
  entities by one of the rules above.  Attendant entities allow type checking 
  of direct member selection of an object even if that object's type isn't exported.
  Declarations, such as \grammarterm{asm-declaration} or
  \grammarterm{alias-declaration} or 
  \grammarterm{static\_assert-declaration},
   that do not declare entities do not contribute to the set of attendant entities.
  \exitnote
  \begin{example}
    \begin{codeblock}
      export module M;
      export struct Foo;          // Foo exported as incomplete type
      struct Foo { };
      export using ::Foo;         // OK: exports complete type Foo

      struct C { };
      struct S {
        struct B { };
        using C = ::C;
        int i : 8;
        double d { };
      };

      export S f();       // S attendant entity of f().

      // translation unit 2
      import M;
      int main() {
        int x = sizeof(decltype(f())::B);   // error: incomplete B
        int y = sizeof(decltype(f())::C);   // error: incomplete C
        decltype(f()) s { };
        s.d = 3.14;                         // OK
        return &s.i != nullptr;             // error: cannot take address of bitfield
      }
    \end{codeblock}
  \end{example}

  \alinea
  If $X$ is an attendant entity of two exported declarations
  designating two distinct entities, its reachable semantic properties shall
  be the same at the points where the declarations occur.
  \begin{example}
    \begin{codeblock}
      export module M;
      struct S;
      export S f();     // \#1
      struct S { };
      export S g();     // error: class type S has different properties from \#1
    \end{codeblock}
  \end{example}
  
\end{std.txt}

